\section{Methodology}
To be able to begin to work on a software first a plan must be designed, what the software should be able to achieve, by designing a process which can recognise a basic set of instructions from text. This instruction set the beginning should be basic and could be further extended as the project requires it. I am not planning to use the full set of instructions used in practice because they are reliant on personal human interpretation and are mostly lacking specific formalities, also by using a smaller more meaningful and descriptive instruction set makes the application more robust. The first set is constructed of the instructions of “begin” and “end” point of the route which are usually defined at the beginning and end of the instruction list. I will also define basic direction rules such as “turn”, “keep”, “left”, “right” and distance measures “X meters”. Another important feature is the extraction of street and landmark names which give a well-defined point in space where geo-locations can be drawn to. Other more specific actions can be added later (“roundabout”, “next”, “exit”, “enter” …) expanding the complexity of the system further. The software than must be able to turn these instructions into gps coordinates and based on gps coordinates it can generate natural text, which preferably must be identical to the textual instructions first given to the system. 

\subsection{Resources}
Based on various research papers and previous expertise on this field the software is the easiest written by the Python programming language which gives the opportunity to make advantage of the open-source Python libraries such as NLTK and Spacy, where natural language processing functions are already introduced, thus making the implementation of the study easier. For the geo-location implementation, the OpenStreetMap API could be used. Resources for the last mile instructions will be drawn from various places as the study extends, first they will be constructed based on the pre-defined basic rule-set after that they will be extended the instruction used by a well-established company on the field, and other openly available last mile route descriptions. 

\subsection{Research question 1}
For the design for the task of creation geolocations from the description with natural language interpretation the following system will be used. The text has to go through a set of pre-processing steps to achieve the normalisation of the text for this preferably tokenisation and lemmatisation will be used. After this the next task is the construction of the semantic parser to convert the natural language to a machine-understandable representation. In this part two distinct types of instruction could be distinguished 1. where Instruction are given with landmark and 2. where instruction are not containing landmark or distance. For the first possibility an estimate of the turning point is calculated in that which is used to collect landmarks in margin area and most promising landmark as point of interest.  For the second types of inctruction first an estimate of the possible turning points has to be calculated. The research than has to define a rule to choose the most promising location. These points can be later added as coordinates to the route. The output of the software for this part will only generate these coordinates. 

\subsection{Research question 2}
For the second function of the system the geo-locations has to go through a few pre-processing steps, first finding the corresponding landmarks and then constructs enriched objects. If a valid landmark can be considered the system should use it automatically if there could be no valid landmark found than distance and direction from the previous point should be calculate and rounded. Between each point the possible instructions have to weighted and sorted (“turn” or “keep”, “left” or “right” …) based on this information an enriched route and instruction set can be constructed which can be used as an input to natural language generation. For this purpose, at the beginning two approaches will be simultaneously used and the more fitting one will be chosen in the final design. The first one a basic natural language generator which only alters the input data to text format with only minor changes and the second one a template-driven solution which could use a Markov-chain based method as a stochastic model potentially providing fast and memory-light solution.
