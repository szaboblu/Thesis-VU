\section{Evaluation}

In order to fully assess the design research process an evaluation of the created design has to be conducted, for this purpose a simulation testing together with an analytical evaluation of the system will be implemented. The testing will be constructed in stages, on each stage the complexity of the instruction set as well as the NLP and NLG tools will be increased this way giving a fuller picture of the development in the software design. Each layer will be lead as follows: First an instructions set is constructed of the basic instructions, other more specific actions will be added on each step expanding the complexity of the system further. Using these instruction set a dataset containing at least 20 direction instruction will be constructed depending on the number of possible instruction variants, which can be constructed using the instruction set. At this stage the tester will compare the workings of the system with manual geo-track translation, they will record the steps on a simple offline map and compare it with the machine generated routes. These two results will be scored on different aspects such as precision, route usability, Geographical positioning and length of track. These scores than will be analysed at the end of the testing phase. Depending on whether the second part of the research question can be answered and this part of the software constructed a second part of the assessment will commence using the output of the first part as an input. The results will then be compared with the original instructions using Jaccard similarity measures.