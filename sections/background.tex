\section{Related literature}

\subsection{Natural language processing }

Natural language processing (NLP) is a subfield of linguistics, computer science, concerned with the interactions between computers and human language. The goal is a software capable of understanding the contents of documents. Natural language processing has its roots in the 1950s when the first rule based NLP methods were developed, these systems were based on complex sets of hand-written rules \cite{guida1986evaluation}. Despite the popularity of machine learning in NLP research, symbolic methods are still commonly used when the amount of training data is insufficient to successfully apply machine learning methods for tokenization, which is the process of demarcating and possibly classifying sections of a string of input characters  \cite{hutchins2005history}. As a result, natural language generation (NLG), a software approach for generating natural language output, was created.  
 \subsection{Interpreting navigation instuctions using NLP}

	The capacity to read instructions and conduct the appropriate actions is critical for seamless interactions with a computer or a robot, and some recent work has studied ways to translate natural-language instructions into actions that can be done by a computer \cite{lau2009interpreting} \cite{branavan2009reinforcement}. And in particular on the task of navigation which is the main focus of this thesis \cite{macmahon2006walk}. Current research in this topic focuses mostly on machine learning applications due to the interest in this field and the possible answers it may give \cite{hashimoto2016joint}. However, rule-based solutions are also observable in the field, partly because these instructions are already well standardised, albeit this differs amongst instruction set suppliers, and constructing a big training set might be problematic. The goal of the navigation task is to take a set of natural language instructions, convert them into a navigation plan that the computer can understand, and then display that path in some way. The definition of Route direction, a special type of instruction that describes how to get from one location to another, extends this definition \cite{wunderlich1982get}.\\
	
The development of systems that learn to interpret navigation instructions has lately gotten a lot of interest because of its potential application in the development of mobile robots. For example, Shimizu and Haas built a system that learns to read navigation instructions by restricting the number of operations to 15 and approaching the task as a sequence labeling problem \cite{shimizu2009learning}. Vogel and Jurafsky built a learning system that uses reinforcement learning to learn to navigate from one landmark to another \cite{vogel2010learning}.
