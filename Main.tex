% This is samplepaper.tex, a sample chapter demonstrating the
% LLNCS macro package for Springer Computer Science proceedings;
% Version 2.20 of 2017/10/04
%
\documentclass[runningheads]{llncs}
%
\usepackage{graphicx}
\usepackage{hyperref}

\usepackage{biblatex}
\addbibresource{references.bib}
% Used for displaying a sample figure. If possible, figure files should
% be included in EPS format.
%
% If you use the hyperref package, please uncomment the following line
% to display URLs in blue roman font according to Springer's eBook style:
% \renewcommand\UrlFont{\color{blue}\rmfamily}

\begin{document}
%
\title{Developing a general model for automated natural language acquisition for navigational instructions }
%
%\titlerunning{Abbreviated paper title}
% If the paper title is too long for the running head, you can set
% an abbreviated paper title here
%
\author{Balázs Szabó \\ \email{\texttt{b.szabo@student.vu.nl}}}
%
%\authorrunning{F. Author et al.}
% First names are abbreviated in the running head.
% If there are more than two authors, 'et al.' is used.
%
\institute{Vrije Universiteit Amsterdam,\\ Amsterdam,
The Netherlands}
%Springer Heidelberg, Tiergartenstr. 17, 69121 Heidelberg, Germany
%\email{lncs@springer.com}\\
%\url{http://www.springer.com/gp/computer-science/lncs} \and
%ABC Institute, Rupert-Karls-University Heidelberg, Heidelberg, Germany\\

%\email{\{abc,lncs\}@uni-heidelberg.de}
%
\maketitle              % typeset the header of the contribution
%
\begin{abstract}


\keywords{First keyword  \and Second keyword \and Another keyword.}
\end{abstract}

\section{Introduction}


\subsection{Motivation}

There are several navigational options available nowadays. We may use maps or GPS devices, but we must always be able to interpret instructions. In the transport area, shippers offer textual instructions for truck drivers on how to get to a retail location for store replenishment. Instructions are prepared for board computers to follow in order to travel to the shop along a predetermined path. Navigation-based and task-oriented conversation systems are commonly used to connect users to maps and navigation software \cite{williams2017hybrid}. There is need for a software which can interpret these directions to geo-locations and is also able to construct descriptions of defined routes. Current research focuses primarily on machine learning applications in this field, due to the interest it has generated and the potential solutions it may provide \cite{hashimoto2016joint}. However, other options must always be considered, primarily since these instructions are already highly standardised, though this varies between instruction set providers, and constructing a large training set could be difficult. Rule- based solutions must also be examined for the benefit of simplicity and the fact that they are already well established in the domain \cite{xu2021ontology}. In this research I am focusing on the development of a software for the need of this industry which can create geolocations from the description, instructions defined as last mile route from begin to end point, with natural language interpretation and is able to create descriptive text based on the defined last mile route as geo-locations.

\subsection{Problem definition }

This paper attempts to solve the problem which accures when trying to translate written direction instructions to machine undestandable format. The problem will be considered solved once a software is designed and tested which is able to recognise a basic set of direction instructions from text, which method should be based on a possible multilingual approach. Can construct a list of valid gps locations from it, and can translate these gps locations into textual instructions.  

\subsection{Research questions  }

The main focus of this thesis is on the design process of a possible software which fulfills the aformentioned cryteria and to answer the main research question of the thesis:  \\

 \textbf{Which software components are needed to integrate natural language processing in the interpretation and generation of navigation instructions for truck drivers?  }\\

 
To be able to answer the main research question, the following sub-questions first need to be answered: 
\begin{itemize}
    \item [] \textbf{Research question 1}: \textit{How can we develop a natural language interpretation software convert written navigational instructions into coordinates?} 
    \begin{itemize}
        \item Research Strategy 1: Design science research to develop an NLP model for navigational instructions interpretation into conceptual representation of the route. 
        \item Research Strategy 2: Develop a model for mapping textual location references from conceptual representation of the route to GPS coordinates. \\
 
    \end{itemize}
    \item[] \textbf{Research question 2}: \textit{How can we develop a natural language generation software to convert a list of coordinates into textual navigational instructions?} 
\begin{itemize}
    \item Research Strategy: Design science research to develop an NLG model for navigational instructions. 
 
\end{itemize}
\end{itemize}
\\

To retrieve information about the feasibility of the solutions found during RQ 1 and 2, and answer the main research question an evaluation of the results will be conducted using a simulation and analytical measures. 


\subsection{Scientific and practical contribution  }

The present focus of research in the subject of automatic natural language acquisition for navigational instructions is mostly on the capabilities that may be used to the navigational aspects of developing autonomous robots \cite{chen2011learning}. This research aims to build on previous work by concentrating on human-machine interactions and developing software that combines NLP and NLG for a more comprehensive language acquisition mechanism. 
\section{Related literature}

\subsection{Natural language processing }

Natural language processing (NLP) is a subfield of linguistics, computer science, concerned with the interactions between computers and human language. The goal is a software capable of understanding the contents of documents. Natural language processing has its roots in the 1950s when the first rule based NLP methods were developed, these systems were based on complex sets of hand-written rules \cite{guida1986evaluation}. Despite the popularity of machine learning in NLP research, symbolic methods are still commonly used when the amount of training data is insufficient to successfully apply machine learning methods for tokenization, which is the process of demarcating and possibly classifying sections of a string of input characters  \cite{hutchins2005history}. As a result, natural language generation (NLG), a software approach for generating natural language output, was created.  
 \subsection{Interpreting navigation instuctions using NLP}

	The capacity to read instructions and conduct the appropriate actions is critical for seamless interactions with a computer or a robot, and some recent work has studied ways to translate natural-language instructions into actions that can be done by a computer \cite{lau2009interpreting} \cite{branavan2009reinforcement}. And in particular on the task of navigation which is the main focus of this thesis \cite{macmahon2006walk}. Current research in this topic focuses mostly on machine learning applications due to the interest in this field and the possible answers it may give \cite{hashimoto2016joint}. However, rule-based solutions are also observable in the field, partly because these instructions are already well standardised, albeit this differs amongst instruction set suppliers, and constructing a big training set might be problematic. The goal of the navigation task is to take a set of natural language instructions, convert them into a navigation plan that the computer can understand, and then display that path in some way. The definition of Route direction, a special type of instruction that describes how to get from one location to another, extends this definition \cite{wunderlich1982get}.\\
	
The development of systems that learn to interpret navigation instructions has lately gotten a lot of interest because of its potential application in the development of mobile robots. For example, Shimizu and Haas built a system that learns to read navigation instructions by restricting the number of operations to 15 and approaching the task as a sequence labeling problem \cite{shimizu2009learning}. Vogel and Jurafsky built a learning system that uses reinforcement learning to learn to navigate from one landmark to another \cite{vogel2010learning}.

\section{Methodology}
To be able to begin to work on a software first a plan must be designed, what the software should be able to achieve, by designing a process which can recognise a basic set of instructions from text. This instruction set the beginning should be basic and could be further extended as the project requires it. I am not planning to use the full set of instructions used in practice because they are reliant on personal human interpretation and are mostly lacking specific formalities, also by using a smaller more meaningful and descriptive instruction set makes the application more robust. The first set is constructed of the instructions of “begin” and “end” point of the route which are usually defined at the beginning and end of the instruction list. I will also define basic direction rules such as “turn”, “keep”, “left”, “right” and distance measures “X meters”. Another important feature is the extraction of street and landmark names which give a well-defined point in space where geo-locations can be drawn to. Other more specific actions can be added later (“roundabout”, “next”, “exit”, “enter” …) expanding the complexity of the system further. The software than must be able to turn these instructions into gps coordinates and based on gps coordinates it can generate natural text, which preferably must be identical to the textual instructions first given to the system. 

\subsection{Resources}
Based on various research papers and previous expertise on this field the software is the easiest written by the Python programming language which gives the opportunity to make advantage of the open-source Python libraries such as NLTK and Spacy, where natural language processing functions are already introduced, thus making the implementation of the study easier. For the geo-location implementation, the OpenStreetMap API could be used. Resources for the last mile instructions will be drawn from various places as the study extends, first they will be constructed based on the pre-defined basic rule-set after that they will be extended the instruction used by a well-established company on the field, and other openly available last mile route descriptions. 

\subsection{Research question 1}
For the design for the task of creation geolocations from the description with natural language interpretation the following system will be used. The text has to go through a set of pre-processing steps to achieve the normalisation of the text for this preferably tokenisation and lemmatisation will be used. After this the next task is the construction of the semantic parser to convert the natural language to a machine-understandable representation. In this part two distinct types of instruction could be distinguished 1. where Instruction are given with landmark and 2. where instruction are not containing landmark or distance. For the first possibility an estimate of the turning point is calculated in that which is used to collect landmarks in margin area and most promising landmark as point of interest.  For the second types of inctruction first an estimate of the possible turning points has to be calculated. The research than has to define a rule to choose the most promising location. These points can be later added as coordinates to the route. The output of the software for this part will only generate these coordinates. 

\subsection{Research question 2}
For the second function of the system the geo-locations has to go through a few pre-processing steps, first finding the corresponding landmarks and then constructs enriched objects. If a valid landmark can be considered the system should use it automatically if there could be no valid landmark found than distance and direction from the previous point should be calculate and rounded. Between each point the possible instructions have to weighted and sorted (“turn” or “keep”, “left” or “right” …) based on this information an enriched route and instruction set can be constructed which can be used as an input to natural language generation. For this purpose, at the beginning two approaches will be simultaneously used and the more fitting one will be chosen in the final design. The first one a basic natural language generator which only alters the input data to text format with only minor changes and the second one a template-driven solution which could use a Markov-chain based method as a stochastic model potentially providing fast and memory-light solution.

\section{Evaluation}

In order to fully assess the design research process an evaluation of the created design has to be conducted, for this purpose a simulation testing together with an analytical evaluation of the system will be implemented. The testing will be constructed in stages, on each stage the complexity of the instruction set as well as the NLP and NLG tools will be increased this way giving a fuller picture of the development in the software design. Each layer will be lead as follows: First an instructions set is constructed of the basic instructions, other more specific actions will be added on each step expanding the complexity of the system further. Using these instruction set a dataset containing at least 20 direction instruction will be constructed depending on the number of possible instruction variants, which can be constructed using the instruction set. At this stage the tester will compare the workings of the system with manual geo-track translation, they will record the steps on a simple offline map and compare it with the machine generated routes. These two results will be scored on different aspects such as precision, route usability, Geographical positioning and length of track. These scores than will be analysed at the end of the testing phase. Depending on whether the second part of the research question can be answered and this part of the software constructed a second part of the assessment will commence using the output of the first part as an input. The results will then be compared with the original instructions using Jaccard similarity measures.
\input{sections/discussion}
\section{Conclusion}

\printbibliography

\end{document}
